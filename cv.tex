% You may use this as the basis for your own CV, but not for any other
% purpose.

\documentclass[a4paper]{barrucadu-cv}

\firstname{Michael}
\familyname{Walker}
\address{York}{North Yorkshire}{England}
\phone[mobile]{(+44) (0) 7966875255}
\email{mike@barrucadu.co.uk}
\homepage{www.barrucadu.co.uk}
\social[linkedin]{barrucadu}
\social[twitter]{barrucadu}
\social[github]{barrucadu}
\extrainfo{GPG: 7F5A95FD}

\begin{document}

\section{Education}

\subsection{University of York}

\degree{2014–now}
  {Computer Science (Ph.D)}
  {}
  {Expected to submit in December 2017.}

\degree{2010–2014}
  {Computer Systems and Software Engineering (M.Eng)}
  {achieved 2:1}
  {Received first-class honours for dissertation on the formal
    verification of stop-the-world garbage collectors.}

\section{Experience}

\employed[Pusher]{May–Aug 2016}
  {Software Engineering Intern}
  {London}
  {\begin{tightitemize}
    \item Contributed to the implementation of a distributed system
      using \keyword{Go} and \keyword{Raft consensus}:
      \begin{itemize}
      \item Rewrote the storage subsystem, improving benchmark
        performance by a factor of 30.
      \item Fixed bugs and implemented missing functionality, bringing
        error rate down from 80\% to 0\%.
      \item Integrated with \keyword{Stagger}, an open-source,
        in-house, metric reporting service.
      \item Reduced maximum garbage collector pause time by a factor
        of 200.
        \item Implemented \keyword{m-of-n sharding}.
      \end{itemize}
    \item Implemented a tool for \keyword{fuzz testing} arbitrary Go
      interfaces (\textbf{\githublink{pusher}{go-interface-fuzzer}}):
      \begin{itemize}
      \item Behaviour is compared against a reference implementation
        and programmer-specified invariants.
      \item Regression testing can be performed by using a prior
        version as the ``reference'' implementation.
      \end{itemize}
    \end{tightitemize}}

\employed[University of York]{Oct 2014–now}
  {Research Student}
  {York}
  {\begin{tightitemize}
    \item Researcher in the Programming Languages and Systems group,
      investigating the testing of \keyword{concurrent programs},
      using \keyword{Haskell}.
    \item Wrote \keyword{D\'ej\`a Fu}, a concurrency testing library
      (\textbf{\githublink{barrucadu}{dejafu}}):
      \begin{itemize}
      \item Allows the deterministic and systematic testing of
        concurrent Haskell programs.
      \item Supports almost all of standard Haskell concurrency.
      \item Has realistic execution semantics, including an
        implementation of the \keyword{x86} / \keyword{x86\_64}
        relaxed memory model.
      \item Resulted in one peer-reviewed publication.
      \end{itemize}
    \end{tightitemize}}

\employed[CoreFiling]{Jul–Sep 2014}
  {Software Engineering Intern}
  {Oxford}
  {\begin{tightitemize}
    \item Refactored an extensively-used in-house wiki program, fixing
      numerous long-standing bugs.
    \item Built a parser/renderer for a Creole-like markup language
      using \keyword{ANTLR} in \keyword{Java}.
    \item Wrote a \keyword{JIRA} plug-in enabling the use of this
      markup in issue descriptions and comments.
    \end{tightitemize}}

\section{Open Source}

\contribution[logdb]{Aug 2016}
  {Sole Developer}
  {\githublink{barrucadu}{logdb}}
  {Wrote a \keyword{Go} general-purpose, fast, log-structured
    database. It ensures the consistency of data written to disk in
    the event of unexpected termination, and can be used as a storage
    backend for the \keyword{hashicorp/raft} library.}

\contribution[Arch Hurd]{2010–2015}
  {Project Leader}
  {\httplink{www.archhurd.org}}
  {Managed a small, geographically-diverse, development team porting
    core \keyword{Linux} software to the \keyword{GNU/Hurd} platform.
    Also produced installation media and maintained a website and
    online software repository.}

\contribution[Uzbl]{2009}
  {Developer}
  {\httplink{www.uzbl.org}}
  {Was part of a small development team that implemented early-stage
    functionality in an open-source web browser using \keyword{C} and
    \keyword{git}.}

\section{Publications}

\published{2015}
  {Déjà Fu: A Concurrency Testing Library for Haskell}
  {ACM SIGPLAN}
  {Michael Walker and Colin Runciman}
  {ACM SIGPLAN Symposium on Haskell}

\end{document}
