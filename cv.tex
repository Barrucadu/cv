\documentclass[a4paper]{barrucadu-cv}

\newcommand{\range}[2]{#1–#2}
\newcommand{\orange}[1]{\range{#1}{now}}

\firstname{Michael}
\familyname{Walker}
\address{London}{England}
\email{mike@barrucadu.co.uk}
\homepage{www.barrucadu.co.uk}
\linkedin{barrucadu}
\twitter{barrucadu}
\github{barrucadu}

\begin{document}

\section{Experience}

\subsection{Government Digital Service}

\begin{cventry}{\orange{Jun~2019}}{Senior Developer}
  \begin{tightitemize}
  \item Tech lead on a small multidisciplinary \keyword{agile} team
    doing discovery work into GOV.UK Accounts and Personalisation, where I:

    \begin{itemize}
    \item Worked with product and delivery managers to prioritise and delegate work.
    \item Implemented authentication and authorisation with \keyword{OAuth} / \keyword{OpenID Connect}.
    \item Established good practices like \keyword{continuous deployment} and \keyword{service level objectives}.
    \item Proposed and built consensus for large changes to the GOV.UK stack.
    \item Worked with architects and our CDN provider to plan how to transition GOV.UK from a mostly static and very cacheable website to being much more dynamic.
    \end{itemize}

  \item Prototyped and then productionised a machine learning pipeline
    for search result ranking using \keyword{Amazon SageMaker}.

  \item Lead an upgrade from \keyword{Elasticsearch 5} to
    \keyword{Elasticsearch 6}.

  \item Worked with performance analysts to plan and implement
    \keyword{A/B tests}.

  \item Worked with non-technical stakeholders to explain and resolve
    technical constraints.

  \item Line managed a mid-level developer.
  \end{tightitemize}
\end{cventry}

\begin{cventry}{\range{Apr~2018}{Jun~2019}}{Developer}
  \begin{tightitemize}
  \item Made various improvements to the GOV.UK stack: for
    performance, technical debt, and architecture debt.  Mostly
    \keyword{Ruby} and \keyword{Rails} or \keyword{Sinatra}, some
    \keyword{Python}, various types of database
    (e.g. \keyword{PostgreSQL} and \keyword{MySQL}).  All running on
    \keyword{Linux}.

  \item Lead initial experiments into \keyword{load testing}.

  \item Planned and implemented an upgrade from \keyword{Elasticsearch
    2} to \keyword{Elasticsearch 5}.

  \item Worked with \keyword{AWS} infrastructure using
    \keyword{Terraform} and \keyword{Puppet}.

  \item Gave regular support to teams which did not merit a full-time
    developer.
  \end{tightitemize}
\end{cventry}

\subsection{Overleaf}

\begin{cventry}{\range{Jan}{Mar}~2018}{Software Engineer (Part Time)}
  \begin{tightitemize}
  \item Maintained legacy \keyword{Rails} and \keyword{Java} /
    \keyword{JGit} services during the merger with ShareLaTeX.
  \end{tightitemize}
\end{cventry}

\begin{cventry}{\range{Jul}{Sep}~2017}{Software Engineering Intern}
  \begin{tightitemize}
  \item Fixed security issues, ranging from \keyword{CSRF} and
    \keyword{XSS} vulnerabilities to a bug in Heroku's router.

  \item Ported a large \keyword{Rails 4} application to \keyword{Rails
    5}.

  \item Designed and implemented a distributed message bus using
    \keyword{Node} and \keyword{Redis}.
  \end{tightitemize}
\end{cventry}

\subsection{Pusher}

\begin{cventry}{\range{May}{Aug}~2016}{Software Engineering Intern}
  \begin{tightitemize}
  \item Contributed to the productionisation of a prototype
    distributed message bus using \keyword{Go} and \keyword{Raft
      consensus}, where I:

    \begin{itemize}
    \item Used profiling and algorithmic analysis to implement a new
      storage subsystem.
    \item Integrated with an in-house metric reporting service.
    \item Used profiling to reduce maximum garbage collector pause
      time by a factor of 200.
    \item Implemented m-of-n sharding.
    \end{itemize}

  \item Implemented a tool for \keyword{fuzz testing} Go interfaces.
  \end{tightitemize}
\end{cventry}

\subsection{CoreFiling}

\begin{cventry}{\range{Jul}{Sep}~2014}{Software Engineering Intern}
  \begin{tightitemize}
  \item Refactored a \keyword{Java} in-house wiki program, fixing
    numerous long-standing bugs.
  \end{tightitemize}
\end{cventry}

\section{Education}

\subsection{University of York}

\begin{cventry}{\range{2014}{2019}}{Ph.D in Computer Science}
  My thesis ``Revealing Behaviours of Concurrent Functional Programs
  by Systematic Testing'' examined the deterministic testing of
  concurrent programs with shared memory and message passing.
  Supervised by Colin Runciman, and examined by Simon Peyton Jones and
  Ana Cavalcanti.
\end{cventry}

\begin{cventry}{\range{2010}{2014}}{M.Eng in Computer Systems and Software Engineering}
  My dissertation on the formal verification of stop-the-world garbage
  collectors received first-class honours.  Overall, I achieved a 2:1.
\end{cventry}

\section{Publications}

\def\entryheadinglevel{subsection}

\phdthesis{2018}
  {Revealing Behaviours of Concurrent Functional Programs by Systematic Testing}
  {University of York}

\published[10.1007/978-3-319-90686-7\_17]
  {2018}
  {Cheap Remarks about Concurrent Programs}
  {ACM SIGPLAN}
  {Michael Walker and Colin Runciman}
  {ACM SIGPLAN Symposium on Functional and Logic Programming}

\published[10.1145/2887747.2804306]
  {2015}
  {Déjà Fu: A Concurrency Testing Library for Haskell}
  {ACM SIGPLAN}
  {Michael Walker and Colin Runciman}
  {ACM SIGPLAN Symposium on Haskell}

\textit{PDFs and BibTeX entries are available on \httplink[my website]{www.barrucadu.co.uk}.}

\section{Open Source}

\begin{cventry}{\orange{2015}}{D\'{e}j\`{a}~Fu}
  A library for testing concurrent \keyword{Haskell} programs,
  developed as part of my Ph.D thesis. It has had some commercial users.

  \textbf{\small\githublink{barrucadu}{dejafu}}
\end{cventry}

\begin{cventry}{\range{2010}{2015}}{Arch Hurd}
  A port of the \keyword{Arch Linux} platform to the
  \keyword{GNU/Hurd} kernel.

  \textbf{\small web: \httplink{www.archhurd.org}}
\end{cventry}

\begin{cventry}{2009}{Uzbl}
  A small web browser written in \keyword{C} following the UNIX
  philosophy of ``do one thing well''.

  \textbf{\small web: \httplink{www.uzbl.org}}
\end{cventry}

\end{document}
