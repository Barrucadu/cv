% You may use this as the basis for your own CV, but not for any other
% purpose.

\documentclass[a4paper]{barrucadu-cv}

\newcommand{\range}[2]{#1–#2}
\newcommand{\orange}[1]{\range{#1}{now}}

\firstname{Michael}
\familyname{Walker}
\address{York}{North Yorkshire}{England}
\phone[mobile]{(+44) (0) 7966875255}
\email{mike@barrucadu.co.uk}
\homepage{www.barrucadu.co.uk}
\social[linkedin]{barrucadu}
\social[twitter]{barrucadu}
\social[github]{barrucadu}
\extrainfo{GPG: 7F5A95FD}

\begin{document}

\section{Experience}

\entry[University of York]{\range{2014}{2017}}
  {Postgraduate Teaching Assistant}
  {\begin{tightitemize}
    \item Marked assessments and assisted students during practical
      sessions.  Modules:
      \begin{itemize}
      \item Mathematical Foundations of Computer Science (1st year)
      \item Theory and Practice of Programming (1st year)
      \item Implementation of Programming Languages (2nd year)
      \item Software Testing (masters)
      \end{itemize}
    \end{tightitemize}}

\entry[Overleaf]{\range{Jul}{Sep}~2017}
  {Software Engineering Intern}
  {\begin{tightitemize}
    \item Fixed numerous security issues, ranging from \keyword{CSRF}
      and \keyword{XSS} vulnerabilities to working around a bug in
      Heroku's router.
    \item Ported a large \keyword{Rails 4} application to
      \keyword{Rails 5}.
    \item Designed and implemented a distributed message bus using
      \keyword{Node} and \keyword{Redis}.
    \item Analysed the Overleaf and ShareLaTeX user and subscription
      data to help inform the business model of the post-merger
      company.
    \end{tightitemize}}

\entry[Pusher]{\range{May}{Aug}~2016}
  {Software Engineering Intern}
  {\begin{tightitemize}
    \item Contributed to the implementation of a distributed system
      using \keyword{Go} and \keyword{Raft consensus}:
      \begin{itemize}
      \item Rewrote the storage subsystem, improving benchmark
        performance by a factor of 30.
      \item Integrated with \keyword{Stagger}, an open-source,
        in-house, metric reporting service.
      \item Reduced maximum garbage collector pause time by a factor
        of 200.
        \item Implemented \keyword{m-of-n sharding}.
      \end{itemize}
    \item Implemented a tool for \keyword{fuzz testing} Go interfaces
      (\textbf{\githublink{pusher}{go-interface-fuzzer}})
    \end{tightitemize}}

\entry[CoreFiling]{\range{Jul}{Sep}~2014}
  {Software Engineering Intern}
  {\begin{tightitemize}
    \item Refactored an extensively used in-house wiki program, fixing
      numerous long-standing bugs.
    \item Built a parser/renderer for a Creole-like markup language
      using \keyword{ANTLR} in \keyword{Java}.
    \item Wrote a \keyword{JIRA} plug-in enabling the use of this
      markup in issue descriptions and comments.
    \end{tightitemize}}

\section{Education}

\subsection{University of York}

\entry{\orange{2014}}
  {Computer Science (Ph.D)}
  {Expected to submit in March 2018.}

\entry{\range{2010}{2014}}
  {Computer Systems and Software Engineering (M.Eng)}
  {Received first-class honours for dissertation on the formal
    verification of stop-the-world garbage collectors.  Achieved 2:1.}

\section{Publications}

\published{2015}
  {Déjà Fu: A Concurrency Testing Library for Haskell}
  {ACM SIGPLAN}
  {Michael Walker and Colin Runciman}
  {ACM SIGPLAN Symposium on Haskell}

\pagebreak
\section*{Open Source}

\smallentry{D\'{e}j\`{a}~Fu}{\orange{2015}}{\githublink{barrucadu}{dejafu}}
  {A library for testing concurrent \keyword{Haskell} programs. Has some
    commercial users.}

\smallentry{CoCo}{\orange{2017}}{\githublink{barrucadu}{coco}}
  {A tool for discovering properties of concurrent Haskell programs with
    shared state.}

\smallentry{logdb}{Aug 2016}{\githublink{barrucadu}{logdb}}
  {A \keyword{Go} general-purpose log-structured database. Can be used
    as a storage backend for the \keyword{hashicorp/raft} library.}

\smallentry{Arch Hurd}{\range{2010}{2015}}{\httplink{www.archhurd.org}}
  {A port of the \keyword{Arch Linux} platform to the \keyword{GNU/Hurd}
    kernel.}

\smallentry{Uzbl}{2009}{\httplink{www.uzbl.org}}
  {A small web browser written in \keyword{C} following the UNIX
    philosophy of ``do one thing well''.}

\end{document}
